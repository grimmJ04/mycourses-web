\documentclass[a4paper,12pt]{article}

\usepackage[magyar]{babel}
\usepackage{t1enc}
\usepackage[T1]{fontenc}
\usepackage[utf8]{inputenc}
\usepackage{caption}
\usepackage{hyperref}
\usepackage{graphicx}
\usepackage{float}
\usepackage{listings}
\usepackage{xcolor}
\usepackage{geometry}
\usepackage{ifthen}
\usepackage{todonotes}
\geometry{
    a4paper,
    %    total={170mm,257mm},
    left=20mm,
    right=20mm,
    top=20mm,
}

\definecolor{forestgreen}{RGB}{34,139,34}

\usepackage{url}

\definecolor{forestgreen}{RGB}{34,139,34}

\lstset { %
    language=Java,
    backgroundcolor=\color{black!5}, % set backgroundcolor
    basicstyle=\footnotesize,% basic font setting
    keywordstyle=\color{blue},
    stringstyle=\color{red},
    commentstyle=\color{forestgreen},
}

% Define right-flushed points
\newcommand{\pont}[1]{\hspace*{\fill}\textbf{(#1 pont)}}

\begin{document}
\pagestyle{empty}
\thispagestyle{empty}

\title{\vspace{-22mm}Alkalmazásfejlesztés - I \\ZH}
\author{\vspace{-2mm}SZTE Szoftverfejlesztés Tanszék}
\date{\vspace{-2mm}\the\year{}. \ifthenelse{\month>2 \AND \month<9} {tavaszi} {őszi} félév}

\maketitle

\section{Általános tudnivalók}

A feladat megoldását \textbf{\href{https://biro.inf.u-szeged.hu/}{Bíró}}-ban kell beadni!
A feladat kézzel lesz kiértékelve, \textbf{de csak az a megoldás ami bíróban hiba nélkül fordul.}
Több forduló beadás esetén a \textbf{legutolsó beadást értékeljük ki!}
A fordításhoz használt rendszer a \textbf{Maven}, melyet egy \textbf{wrapper}-en keresztül használunk, \textbf{\color{red}ezt elő kell állítani}.
A Bíró Linux-on fordít.
A feltöltött feladat \textbf{nem} tartalmazhat \textbf{.class} fájlokat és \textbf{nem} lehet benne \textbf{target} mappa.
A feladatot zip csomagként kell feltölteni, mely tartalmazza a teljes projektet \texttt{mvn clean} utáni állapottal.

A ZH során használható anyag a \\
\url{https://biro.inf.u-szeged.hu/kozos/alkfejl/alkalmazasfejlesztes_kozos.zip}\\
linken letölthető, mely egy zip fájl és tartalmazza a
\begin{itemize}
    \item Oktatási weboldal off-line változatát
    \item JavaFX13 dokumentációt
    \item Adatbázis segédletet
\end{itemize}
\vspace{0.6cm}

\textbf{Technikai részletek}
\begin{itemize}
    \item bíró fordítási parancs: \texttt{./mvnw clean compile}
    \item feltöltés előtti lépés: \texttt{./mvnw clean} vagy az IDE-n belül a teljes projectre kiadott clean.
    \item zip elkészítés: \texttt{zip -r feladat.zip <project-mappa>}
\end{itemize}

\section{Maven segítség}

A biro fordításhoz kell a \texttt{mvnw} bináris, mely a
\begin{verbatim}
    mvn wrapper:wrapper
\end{verbatim}
parancs kiadásával oldható meg.
\\
\textbf{\color{red}Ha ez nem található, akkor a projekt nem fordul, így nem értékelhető a megoldás!}
\\
Fontos, hogy ha \textbf{több modul}ból áll a projekt, akkor a \textbf{legkülső pom} fájl mappájában adjuk ki a Maven-es parancsokat!

\section{Bíró eredmények}

A kiértékelés során 0 vagy 1 pont szerezhető.
Az 1 pont jelzi a sikeres feladatbeadást, mely értékelésre kerül.
A 0 pont sikertelen beadást jelent, melynek az okai a következők lehetnek:
\begin{itemize}
    \item A beadott fájl sérült, nem megfelelő zip fájl vagy sérült zip került feltöltésre.
    \item A beadott zip tartalmaz \textbf{target} mappát.
    \item A beadott zip tartalmaz \textbf{.class} (java bináris) fájlt.
    \item A beadott projekt a fordítás során hibával leállt.
\end{itemize}

A Maven fordítás eredménye a \textit{mvn.compile} fájlban található.
\newpage
